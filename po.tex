\section{Byzantine Fault Tolerance}

Byzantine Fault Tolerance is a kind of fault tolerance especially important in distributed systems such as Peer-to-Peer systems but also common in other types of computer systems.
Byzantine Fault Tolerance tries to solve the so called Byzantine Generals' Problem which is defined as followed: the Byzantine army is split up in multiple divisions 
outside an enemy city. Each division has a leading general which has to decide whether to attack the city or to retreat. The generals communicate to each other via messengers and they all try to
reach a consensus on a plan of action. Some of the generals however may be traitors who try to prevent an agreement. They do this by sending different conflicting messages to the loyal generals. 
Byzantine Fault Tolerance tries to make sure that a small number of such treacherous generals are not able to disrupt a consensus between the majority of the loyal generals.\cite{url:byzantine_general}
Algorithms developed for the purpose of solving this problem in distributed systems where consensus has to be achieved are called consensus algorithms.

\section{Consensus in distributed \\
blockchains}

The idea behind Proof-of type systems is that whoever should be able to add the next block in the blockchain is chosen by proving the ownership of some kind of resource. This resource could be
something determined by computer hardware, like computing power in Proof-of-Work systems or hard drive capacity in Proof-of-Space systems, or it could be determined by some kind of stake ownership
in the system like in Proof-of-Stake systems. Below we will describe some of these systems and compare them to each other.

\subsection{Proof-of-Work}

The most commonly used consensus algorithm in blockchains is a so called Proof-of-Work (POW) algorithm. It was and is still being used as the consensus algorithm in the implementation of Bitcoin,\cite{url:bitcoin}
and other widely used cryptocurrencies like Litecoin and Dogecoin.
The idea behind a POW algorithm is that the creator of the next block is chosen randomly among peers, where the chance of creating a block is proportional to the amount of computing power invested
by the peer. Usually this computing power is provided by solving some kind of mathematical puzzle.

\subsubsection{Requirements for POW systems}

The puzzle in this context is some kind of computationally expensive operation, which is usually based on some cryptographic operation. After finding a solution to the puzzle it is then added to the 
newly created
block by the peer which solved it, in order to proove to the other peers that the correct solution has been found. Therefore it should be computationally easy for the other peers to check the correctness 
of the solution.
The puzzle chosen for a POW system can and should have some properties. Firstly it should be computationally hard enough to ensure that a significant amount of computational power should have 
to be expended to be able to solve it otherwise the purpose would be lost. On the other hand however it should also be computationally easy for others in the system to check the correctness 
of a proposed solution. Additionally it should also be possible to adjust the difficulty of the puzzle. This is very important to make the system future proof and to account for the predicted
exponential increase in computing power over the years.\cite{url:moore_law} Additionally in the context of blockchains the difficulty of the puzzle should also account for a potential
increase in computing power within the network caused by the addition of more peers.\par
Lastly there is also the consideration of the usefulness of the puzzle. In most existing implementations
of POW systems the puzzle which should be solved does not provide any purpose other than to prove the ownerhship of computational power. 
In other words the solution of the puzzle has no value other then finding the creator of the next block
and the expended computing power is could be considered to be wasted.\cite{url:pow_useless} There have been some propositions and implementations for "useful" POW type systems. One example is
the algorithm used in the cryptocurrency Primecoin. Primecoin uses a POW type system where a byproduct of the solution for the puzzle is the discovery of potentially new prime numbers.\cite{url:primecoin}
However this has some disadvantages, for example it is very hard to adjust the difficulty and therefore control the average time it takes to create a new block.

\subsubsection{Examples for existing POW systems}

The first kind of use for a POW type system was called Hashcash\cite{url:hashcash} and was created to combat spam emails and denial of service attacks. The idea of a system like Hashcash was firstly proposed in 1992
by Cynthia Dwork and Noni Naor.\cite{url:pow_email} The concrete proposal for Hascash and its first implementation followed in 1997.\cite{url:hashcash} The basic idea behind Hashcash is that
the sender takes the header of the email he wants to send, appends a 20 bit long number at the end of it and computes the 160 bit long SHA-1 hash value of it. The goal is to find a hash in which the first 20 bits
are set to zero. If this is not the case the appended number is increased by one and the hash is computed again. This is done until a hash with the first 20 bits set to zero is found. The email
is then sent with the number which satisfies this condition appended to the header. The receiver of the email can then simply try to hash the header himself and he can immediately see if the resulting
hash satisfies the conditions. This system ensures that the sender has to expend some amount of computing power by hashing the header a significant amount of times until it fulfills the condition.
For the receiver however the check can be done immediately since he only has to compute the hash once. Ideally for the sender this means that he has to calculate for some amount of time until
he can send out the email which prevents the sending of a lot of potential spam emails at once. This system also accounts for the increase of computing power over the years since the difficulty of
the computation can be increased by increasing the required number of zero bits in the hash. However the only way to adjust the difficulty is by applying a software update, since the difficulty
is currently statically set to 20 bits.\par
The first use of a POW system for the purpose of a consensus algorithms in a blockchain is however the one used for bitcoins. The idea is very similar to the one used in Hashcash as both systems use a
cryptographic hash function (SHA-256 in Bitcoin) with which they try to find a hash with a certain number of leading zeros, depending on the desired difficulty. Instead of hashing the header of a email
however, in Bitcoin the block which should be added to the chain is hashed. Similar to the Hascash system the block contains a number which is increased until a valid hash is found.
Among other things the new block also contains the hash of the previous block in the chain. This is necessary not olny to ensure that the blocks are linked and create a chain but also to make sure that
someone can not try to find the hash for a future block in advance. As with Hashcash the difficulty can be adjusted by adjusting the required number
of leading zeros in the hash. In the most cryptocurreny networks which use POW this difficulty adjusts itself such that the creation time of a new block is more or less constant. In the case
of the Bitcoin network the average creation time is 10 minutes.
This whole system of trying to find a solution and then adding a block to the chain is called mining. To provide an incentive for
people to mine there is a reward of some amount of Bitcoins for whoever finds the next solution and adds the next block to the chain.\cite{url:bitcoin}\par
Another type of POW algorithm used in cryptocurrencies like Litecoin or Dogecoin is the so called scrypt algorithm.\cite{url:litecoin_dogecoin} The basic concept behind scrypt is that in the first 
step a large number of pseudorandom bit strings are generated. The amount of these bit strings is such that it could very well be possible to run into memory limitations while trying to store 
all the strings. In a second step, these strings are then accessed in pseudorandom order and used to generate a key. Since the execution of this algorithm is not only computationally hard but
also has a significant memory requirement it could also be categorised as a Proof-of-Space type algorithm\cite{url:scrypt} which is further explained in section \rec{proof-of-space}.

\subsubsection{Problems with POW systems}

One of the already mentioned problems with using POW as a consensus algorithm is the already mentioned fact that with most implementations the computing power expended cold be considered
to be wasted especially considering the fact that there is a significant electrical power usage if the network reaches a certain size. In the example of the Bitcoin network this escalated to
the point that the power consumption of the entire bitcoin network is estimated to be at around 3 Gigawatts which is equivalent to the power consumption of the entire state of Morocco.\cite{url:btc_power}

Another problem which arouse after Bitcoin started to become popular, which is also a problem with other types of consensus algorithms is the fact that with a big enough investment it is possible
to overpower the rest of the network. Since the reward for mining is a significant amount of Bitcoins, a lot of people tried to earn money by using their PCs to mine. As more and more people joined
the mining network, the difficulty increased and more efficient ways of mining Bitcoins emerged. For this purpose Application Specific Integrated Circuits (ASICs) started to emerge. ASICs are 
basically small circuits made for one special purpose. In the case of Bitcoins this purpose is to calculate SHA256 hashes as fast as possible.\cite{url:asic} These ASICs where a lot cheaper and more
energy efficient then PCs and started to make mining with them unprofitable.
Soon even companies emerged
which had bitcoin mining as their only business model. This happened especially in China, because both the hardware and the electricity cost is comparably very low compared to other countries. This lead
to the fact that today an estimated 71\% of the mining power comes from a few Chinese companies with halls full of dedicated Bitcoin mining hardware. 
This lead to so called hash power centralization and therefore the loss of
one of the core concepts of blockchains, namely the decentralization.\cite{url:btc_china}

\subsection{Proof-of-Stake}

The idea behind a Proof-of-Stake (POS) type system is that instead of determining the peer which should be able to add the next block through computing power, as in POW systems, the peer is instead
selected by calculating some stake value first and then choosing a peer based on the highest stake in the network. There are various ways to calculate the required stake values, each with different
criteria. The trust from other peers in the network then comes from the fact that the peer with the highest stake has the highest incentive add a block which is in agreement with the other peers.
Otherwise, if consensus is lost, then the trust in the network is lost as well and that peer loses the most since he has the highest stake. 
The idea was first proposed by the user QuantumMechanic in a blog post.\cite{url:pos}

\subsubsection{Requirements for POS systems}

Some things have to be considered when trying to implement a POS type system. Firstly there should be a mechanism to ensure that a single peer is not able to create multiple blocks in a row, since
that would give it a control over the chain. Another problem which emerges when trying to implement a POS system is the resolution of so called forks in the blockchain. Forks happen
quite frequently in blockchains, when there is no clear consensus at one point in the network. There has to be a way to resolve such forks and to regain consensus in some way. 
In the case of POW systems this problem is resolved by looking at the combined difficulty of the blocks in one of the forks.
After some amount of time one of the forks will have a higher difficulty
compared to the others and the peers will then adopt the fork with the highest combined difficulty and drop all other forks.\cite{url:bitcoin}
In the case of POS systems however if a fork happens and there is no single consensus, a peer has nothing to loose by trying to follow both forks, which in turn means that regaining
consensus is impossible. Different implementations have different ways of dealing with such a consensus failure.\cite{url:pos_impossible}

\subsubsection{Examples of existing POS systems}

One example of a POS type systems can be found in the implementation of the cryptocurrency Nxt. 
In the Nxt network a peer is mining (often called forging in POS) by taking the generation signature from the previous block in the chain and signing it with its own public key. The resulting
64-byte signature is then hashed using SHA256 and saved as an \textit{account hit}. This account hit is then compared to the peers current target value and if the account hit is lower than the
target value, the peer is validated to create the next block. The current target value is individual for each peer and is calculated using the following formula: $$T=T_b*S*B_e$$ where $T$
is the new current target value of a peer, $T_b$ is a base target value which is constant for each block and calculated such that the creation time of each block remains at a constant 60
seconds, $S$ is the time in seconds since the last block was created and $B_e$ is the account balance of the peer. As we can see the target value increases every second, to ensure that
the probability for a peer to get a account hit lower than the target value increases. The current target value is also higher for peers with a higher account balance such that peers with a higher
stake have a higher chance and we therefore get a POS system. The result of this algorithm is that a peer which owns 5\% of the coins in the network should approximately be able to create 5\% 
of the blocks
in the chain. To ensure that a single peer is unable to create multiple blocks in a row a peer is only allowed to mine if he did not create one of the past 1440 blocks. Additionally
a peer has to have owned its current coins for at least 1440 block before its stake is considered to ensure that a peer does not simply move his stake between wallets and is therefore able to create
multiple blocks in a row. The problem of consensus loss is solved similarly as with POW systems as each block has an assigned difficulty value based on the target value of the creating peer.\cite{url:nxt}\par
Another type of POS system was proposed during the creation of the cryptocurrency Ethereum. The so called Slasher algorithm, which was never implemented in Ethereum, proposes an algorithm in
which a peer that tries to follow two forks in the blockchain is being punished by the network. In short what happens is that if a peer tries to add a block to both forks of the chain, then
other peers which notice this behaviour can publish their finding to the network. The result is then that the \textit{cheating} peer loses its reward for adding the block and the peer who found the
"cheater" gets 33\% of the reward.\cite{url:eth_slash} This algorithm was however discarded by the Ethereum developers, since not all consensus problems could be solved and the implementation of the punishing cheater
algorithm by itself would be non-trivial.\cite{url:eth_no_slash} Ethereum instead now uses a POW type algorithm.\cite{url:eth_pow}\par
Another cryptocurrency which uses POS as a part of its consensus algorithm is Peercoin. Similarly to Nxt in Peercoin a stake is calculated for each peer. In comparison to Nxt however the age
of a coin i.e. the time a coin is in possession of a peer is also relevant for calculating a peers stake. The older a coin is, the higher its value. In the case of consensus failure consensus is then
regained by choosing the fork with the higher number of collective coin age.\cite{url:peercoin} One big criticism for Peercoin however is that it uses a periodical, 
centralized broadcast issued by the developers of the cryptocurrency to checkpoint blocks in the chain.
These checkpointed blocks and all blocks in the chain before it have to be considered valid by all peers. This however causes the whole blockchain to be somewhat centralized.
Most other implementation which use POS algorithms make use of a hybrid between POS and POW algorithms called Proof-of-Activity later described in section \ref{poa}.

\subsection{Proof-of-Activity} \label{poa}

Since both POW and POS type systems show some major disadvantages to each other, the idea of creating a hybrid between the two concepts emerged pretty soon. The idea is to use the properties of POW
systems to resolve consensus failures, while not having to solely rely on computational power to reach consensus.\par
First we have to describe the so called \textit{follow-the-satoshi} subroutine
where a satoshi is the smallest possible unit of a cryptocurrency. Follow-the-satoshi selects a random peer in the network by taking a number between 1 and the total number of
satoshis in the network as input. This satoshi is then followed though the saved transactions in the blockchain until it reaches the current owner which is then the selected owner. In this way the chance
of beeing the selected owner is higher if ones account balance is higher.\par
The Proof-of-Activity (POA) mining protocol then works by walking through the following steps:
\begin{itemize}
    \item A miner tries to find the valid hash value for the next block just like in POW systems and creates a empty block header if such a hash has been found.
    \item The header is then broadcast by the miner to the whole network.
    \item Every peer then derives N stakeholders by concatenating the hash of the previous block with the hash in the new blocks header appending N fixed suffix values and hashing each combination
    resulting in N different values which are then used to find N stakeholders by applying follow-the-satoshi.
    \item Every stakeholder who is online then checks if the new block header is valid and wether he is one of the N derived stakeholders. If this is the case the peer signs the hash of the new block
    with its private key and broadcasts the signature. If the Nth stakeholder is reached it includes some amount of transactions from the current transaction pool to the block as well as its own
    signature.
    \item The Nth stakeholder then broadcasts the block and all other peers can check its validity by checking the validity of the hash in the header, derive all selected stakeholders and check their
    signatures. The reward for mining the block is then shared among the miner which found the valid hash and all stakeholders which helped verifying the block.
\end{itemize}

It is also important to add, that if some of the stakeholders are offline and therefore unable to sign the current block header, then after some time other miners will find different valid hashes
which will then derive different stakeholder until all selected stakeholders where able to add their signature.\cite{url:poa}

\subsection{Proof-of-Capacity or Proof-of-Space}

\label{proof-of-space}

A different kind of proposed consensus algorithm is the so called Proof-of-Capacity (POC), sometimes also referred to as Proof-of-Space, algorithm. The idea of POC is similar to the one used in POW, but
instead of using computational power as proof, disk space is used. This is done in order to reach consensus in a similar way to POW while being more ecologically friendly and
fairer to smaller peers, since disk space is less expensive compared to computational power and there is no danger for designated hardware to emerge for the sole purpose of mining.\par
The basic concept of the POC algorithm is that a peer generates some large data set in a predetermined way and stores it on his disk. If some other peer then wants to verify that the disk space is used
he can ask for some specific section of the generated data set, which the original peer can then provide. For this to work the generation of the data set has to be computationally expensive enough
to ensure that the proving peer cannot generate the data set on the fly and provide the requested subset, but has to actually keep it stored.\cite{url:poc}

\subsubsection{Existing POC systems}

For the most part the usage of POC has only been theorised. The only somewhat large scale implementation of a blockchain with POC as its consensus algorithm is with the cryptocurrency Burstcoin.
In its in 2014 released open implementation mining works in the following way: firstly every potential miner has to generate a large data set by taking his public address and a nonce and then
applying a hash function to it multiple times. The resulting data set is then called a plot. This plot is then divided into 4096 so called scoops and stored on the disk. This process of generating
a plot and storing it only has to be done once for every miner. After that mining works by firstly determining a scoop number which is derived from the previous block. The scoop with this scoop
number is then the only scoop relevant for the next block and is therefore the only scoop that has to be read from the whole dataset which is computationally very easy. The scoop is then hashed
with the generation signature of the next block. 8 bytes from the resulting hash are then taken and divided by a scaling factor, which can be adjusted and serves the same purpose as the difficulty
in POW systems. The result is then a number of seconds. If this number of seconds have passed since the generation of the last block then the user whose scoop was used to generate it is eligible
to add the new block to the chain. This way every user would be eligible if enough time passes.\cite{url:burstcoin}\par
Other POC based cryptocurrencies like SpaceMint have also been proposed with similar algorithms but were never fully implemented. but where never fully implemented.\cite{url:spacemint}

\subsection{Proof-of-Burn}

Proof-of-Burn (POB) has been proposed as an alternative to POW systems. The way POB works is that to be able to start mining a peer has to "burn" some amount of coins first. Burning in this context
means that the coins are being sent to a address from which they are unretrievable and therefore lost forever. The more coins a miner sends to such an address the higher his chance becomes
to be the creator of the next block. The idea is that instead of spending money to buy mining hardware for POW systems, the money is instead directly burned skipping one step in between.
For the purpose of burning coins, coins from other cryptocurrencies can also be used. One major criticism for this concept is however that all current implementations only allow for the
burning of currencies which use the POW system. Which then nullifies the advantage POB would have over POW.\par
The only at least somewhat active cryptocurrency currently using POB is Slimcoin\cite{url:pob}

\subsection{Proof-of-Elapsed-Time}

Proof-of-Elapsed-Time (POET) is another type of consensus algorithm proposed by Intel in 2015 as an alternative to POW. The idea is very similar to the one proposed by POW but instead of having
to expend computing power to solve a puzzle which takes approximately a predetermined amount of time, this time is just spend waiting instead of consuming computing power and therefore
electrical power. This provides some clear advantages. For once it is far more economical in comparison to POW. Secondly it provides a lot more fairness, since spending more money on hardware
does not increase the chance to be the one to validate a block. According to Intel it is also extremely efficient to implement such an algorithm using CPU level instructions already present on
newer generations of Intel CPUs. This however is also one of the major concerns of this proposed algorithm. By having to rely on Intel specific hardware, the whole system designed to be
decentralized becomes very dependent on Intel as a manufacturer, which is in contradiction to the idea of a blockchain in the first place.\cite{url:coinbase_consensus}\par
The basic idea of the functionality of the algorithm is as follows: the peers request a wait time from a so called enclave, which is just a function which is considered trustworthy by all peers
which by Intels idea is present as a CPU instruction. The peer with the lowest wait time is then the one who is validated by the other peers in the network for whatever operation he wants to do,
in the case of a blockchain this would be to create a new block. The other peers have to then be able to check, that the timer the validated peer used as a wait time is a valid timer he got from the
trusted enclave. This can again be done by all peers through the enclave.\par
There is currently no real application using this algorithm, but Intel released an open-source implementation of such an algorithm which can be used for testing.\cite{url:poet}

% Net.In.Tum (Pro)Seminar Paper Template
% Based on http://www.acm.org/sigs/publications/proceedings-templates
% (Option #1)
% 2015-11-04

% +++++++++++++++++++++++++++++++++++++++++++

% ENGLISH-Version
\documentclass{acm_proc_article-sp}

% GERMAN-Version
%\documentclass{acm_proc_article-sp-german}
%\usepackage[ngerman]{babel}
% Be sure to toggle other language relevant settings accordingly below...

% +++++++++++++++++++++++++++++++++++++++++++
% Include packages for additional functionality

\usepackage{makeidx}
\usepackage{graphicx}
\usepackage{tabularx}
\usepackage{amssymb}
\usepackage{amsmath}
\usepackage{color}
\usepackage{hyperref}
\usepackage[utf8]{inputenc}

%+++++++++++++++++++++++++++++++++++++++++++

\begin{document}

\title{Consensus Algorithms for Blockchains}

\numberofauthors{1}
\author{Marcel Mussner\\
       \affaddr{Advisor: Sree Harsha Totakura}\\
       %\affaddr{Betreuer: Name Vorname}\\
       \affaddr{(Pro)Seminar Network Hacking, Defense and Tamperproof Systems WS2017/18}\\
       \affaddr{Chair for Network Architectures and Services}\\
       \affaddr{Department of Computer Science, Technische Universität München}\\
       % \affaddr{Lehrstuhl für Netzarchitekturen und Netzdienste}\\
       % \affaddr{Fakultät für Informatik, Technische Universität München}\\
       \email{Email: marcel.mussner@tum.de}
}

\maketitle

%+++++++++++++++++++++++++++++++++++++++++++

\begin{abstract}


\end{abstract}

%+++++++++++++++++++++++++++++++++++++++++++

%\keywords{Foo, Bar, Baz}

%+++++++++++++++++++++++++++++++++++++++++++

\section{Introduction}

\section{Blockchains}

A blockchain is a collection of records, called blocks, which are linked together forming a chain. These links are created and secured by using different kinds of cryptographic functions.
The first theoretical description of such a cryptographically secured chain was made by Stuart Haber and W. Scott Stornett in 1992.\cite{book:haber} In 2008 a person or group known as Satoshi Nakamoto
described the first practical implementation for a blockchain in their white paper called "Bitcoin: A Peer-to-Peer Electronic Cash System".\cite{url:bitcoin} In this paper a new type of digital
currency, later categorised as cryptocurrency, which uses a blockchains for keeping track of transactions is described in detail. One year later this concept has been implemented and released to the
public. The idea behind these blockchains is that they are part of a network of peers. Within this network the blockchain is public knowledge to everyone and at least the majority of the peers in the
network should be in consensus about a newly added block.

\subsection{Byzantine Fault Tolerance}

Byzantine Fault Tolerance is a kind of fault tolerance especially important in distributed systems such as Peer-to-Peer systems but also common in other types of computer systems.
Byzantine Fault Tolerance tries to solve the so called Byzantine Generals' Problem. The Byzantine Generals' Problem is defined as follows: the Byzantine army is split up in multiple divisions 
outside an enemy city. Each division has a leading general which has to decide whether to attack the city or to retreat. The generals communicate to each other via messengers and they all try to
reach a consensus on a plan of action. Some of the generals however may be traitors who try to prevent an agreement. They do this by sending different conflicting messages to the loyal generals. 
Byzantine Fault Tolerance tries to make sure that a small number of such treacherous generals are not able to disrupt a consensus between the majority of the loyal generals.\cite{url:byzantine_general}
Algorithms developed for the purpose of solving this problem in distributed systems are called consensus algorithms.

\subsection{Blockchains in the example of Bitcoin}

The blockchain is a core component of the way Bitcoin and other later developed cryptocurrencies word. The blockchain is used as a public ledger for all transaction which happened in the Bitcoin
network. For this purpose every block in the chain contains among other things a collection of transactions that happened in a time period. By following the blocks in the chain it is possible to
determine exactly where a Bitcoin (or a fraction of a Bitcoin) came from and who it belongs to now. Since this Blockchain is public knowledge there is no dispute over who a Bitcoin belongs to.
One of the most important things then is who should be able to add a new block to the blockchain such that every peer in the network can agree to this new block. In other words how a consensus between
the peers can be reached. In the case of Bitcoin this is done through a Proof-of-Work algorithm.

\subsection{Advantages and use cases of blockchains}

Explanation why blockchains solve some of the problems of other systems and what types of applications this could be useful for.

\subsection{Problems with blockchains}

Description of some problems and disadvantages of blockchains in general, without looking at specific implementations yet.

\subsection{Existing implementations}

Some examples of systems where blockchains are already used.


\section{Proof-of-Work}

The idea behind Proof-of type systems is that whoever should be able to add the next block in the blockchain is chosen by proving the ownership of some kind of resource. This resource could be
something determined by computer hardware, like computing power in Proof-of-Work systems or hard drive capacity in Proof-of-Space systems, or it could be determined by some kind of stake ownership
in the system like in Proof-of-Stake systems. Below we will describe some of these systems and compare them to each other.

\subsection{Proof-of-Work}

The first widely used consensus algorithm in blockchains is a so called Proof-of-Work (POW) algorithm. It was and is still being used as the consensus algorithm in the implementation of Bitcoin.\cite{url:bitcoin}
The idea behind a POW algorithm is that the creator of the next block is chosen by him providing a significant amount of computing power. Usually this computing power is provided by solving some
kind of puzzle. 

\subsubsection{Requirements for POW systems}

The puzzle chosen for a POW system can and should have some properties. Firstly it should be computationally hard enough to ensure that a significant amount of computational power should have 
to be expended to be able to solve it otherwise the purpose would be lost. On the other hand however it should also be computationally easy for others in the system to check the correctness 
of a proposed solution. Additionally it should also be possible to adjust the difficulty of the puzzle. This is very important to make the system future proof and to account for the predicted
exponential increase in computing power over the years.\cite{url:moore_law} Additionally in the context of blockchains the difficulty of the puzzle should also account for a potential
increase in computing power within the network caused by the addition of more peers.\par
Lastly there is also the consideration of usefulness of the puzzle. In most existing implementations
of POW systems the puzzle which should be solved does not provide a higher purpose. In other words the solution of the puzzle has no value other then finding the creator of the next block
and the expended computing power is could be considered to be wasted.\cite{url:pow_useless} There have been some propositions and implementations for "useful" POW type systems. One example is
the algorithm used in the cryptocurrency Primecoin. Primecoin uses a POW type system where a byproduct of the solution for the puzzle is the discovery of potentially new prime numbers.\cite{url:primecoin}

\subsubsection{Examples for existing POW systems}




\section{Practical Byzantine Fault \\
Tolerance}

Practical Byzantine Fault Tolerance (PBFT) is a algorithm proposed by Miguel Castro and Barbara Liskov in 1999 in order so solve the problem of reaching Byzantine Fault Tolerance in distributed
systems in a way that the performance inpact remains relatively small. The algorithm is a type of state machine replication which offers liveness and safety in a synchronous or
asynchronous system as long as at most $n-1/3$ nodes are simultaneously faulty.\cite{url:pbft}

\subsection{The algorithm}

As already mentioned the algorithm is a form of state machine replication which means the state machine is replicated to different nodes in the system with each replica maintaining bot the current
service state and implementing all available service operations. It is assumed that there are $R=3f+1$ replicas where $f$ is the maximum number of faulty nodes. Each replica is assigned a number
in ${0,...,R-1}$. These replicas then move trough a succession of so called \textit{views} which are just different configurations of the replicas. In each view a replica $p=v\ mod\ R$ is selected
which then represents the \textit{primary} in the current view, while the other replicas represent the \textit{backups}. A view is changed if it is believed that the primary is experiencing a
failure.

The algorithm then works roughly as follows: A client sends a request for the execution of a operation by sending a signed request message to the primary. This message contains
the id of the requested operation, a timestamp and the id of the requesting client. As soon as the primary receives the request it starts a three-phase protocol.
In the first, the pre-prepare phase, the primary multicasts a pre-prepare message to all backups. This message contains a sequence number, the current view, the clients request message
and the hash of the clients request message. The whole message is signed by the primary and it is appended to the primary's log.

In the next step each backup checks the pre-prepare message for its validity by checking if the signature and the hash are correct, checking that the message has the same view as the backup and
that it has not accepted a other pre-prepare message with with the same view and sequence number but a different hash. If a backup \textit{i} accepts a pre-prepare message it enters the \textit{prepare}
phase in which it multicasts a prepare message containing the view, the sequence number and the hash from the pre-prepare message as well as the backups id \textit{i}. Both the pre-prepare and
prepare message are also appended to its log. Other replicas receiving such prepare messages accept them and add them to their log as long as the signature, the hash and the view are
valid. A request is then accepted as a \textit{prepared} message, when a replica receives $2f$ prepare messages from other replicas, which match the pre-prepare message and have valid signatures
and hashes. If no more than $f+1$ replicas are faulty this guarantees that the request is valid and should be accepted. As soon as a replica accepts a prepared message, it is again added to its log
and then multicast in a signed commit message again with the view, the sequence number, the hash and the id of the replica. Afterwards the final commit phase is entered.

Receiving replicas add the commit
message to their log as long as the signature and the hash are valid. The request is then considered to be commited-local by a replica if $2f+1$ commits have been received and accepted from
different replicas (including its own). If a request is considered to be commited-local and all requests with a lower sequence number have been executed, the replica will then execute the operation
requested by the client.

\subsection{Application}

The authors of the paper also included the findings of their attempt to implement the algorithm in a real world application. They implemented the algorithm as a replication library and then used this
library to implement a Byzantine-Fault-tolerant NFS file system. They showed that using the algorithm resulted in a only 3\% time increase in operations on the file system, which proves
that the algorithm could be used for real world applications. 



\section{Conclusions}


%+++++++++++++++++++++++++++++++++++++++++++
% edit literature.bib
%\bibliographystyle{abbrv}
\bibliographystyle{unsrt}
\bibliography{literature}

%+++++++++++++++++++++++++++++++++++++++++++
% APPENDICES are optional
% Enable when needed!
% \appendix

% Appendix #1
%\section{Some Appendix Which is Entirely Optional}

\end{document}

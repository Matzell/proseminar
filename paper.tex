% Net.In.Tum (Pro)Seminar Paper Template
% Based on http://www.acm.org/sigs/publications/proceedings-templates
% (Option #1)
% 2015-11-04

% +++++++++++++++++++++++++++++++++++++++++++

% ENGLISH-Version
\documentclass{acm_proc_article-sp}

% GERMAN-Version
%\documentclass{acm_proc_article-sp-german}
%\usepackage[ngerman]{babel}
% Be sure to toggle other language relevant settings accordingly below...

% +++++++++++++++++++++++++++++++++++++++++++
% Include packages for additional functionality

\usepackage{makeidx}
\usepackage{graphicx}
\usepackage{tabularx}
\usepackage{amssymb}
\usepackage{amsmath}
\usepackage{color}
\usepackage{hyperref}
\usepackage[utf8]{inputenc}

%+++++++++++++++++++++++++++++++++++++++++++

\begin{document}

\title{Consensus Algorithms for Blockchains}

\numberofauthors{1}
\author{Marcel Mussner\\
       \affaddr{Advisor: Sree Harsha Totakura}\\
       %\affaddr{Betreuer: Name Vorname}\\
       \affaddr{(Pro)Seminar Network Hacking, Defense and Tamperproof Systems WS2017/18}\\
       \affaddr{Chair for Network Architectures and Services}\\
       \affaddr{Department of Computer Science, Technische Universität München}\\
       % \affaddr{Lehrstuhl für Netzarchitekturen und Netzdienste}\\
       % \affaddr{Fakultät für Informatik, Technische Universität München}\\
       \email{Email: marcel.mussner@tum.de}
}

\maketitle

%+++++++++++++++++++++++++++++++++++++++++++

\begin{abstract}


\end{abstract}

%+++++++++++++++++++++++++++++++++++++++++++

%\keywords{Foo, Bar, Baz}

%+++++++++++++++++++++++++++++++++++++++++++

\section{Introduction}

\section{Blockchains}

A blockchain is a collection of records, called blocks, which are linked together forming a chain. These links are created and secured by using different kinds of cryptographic functions.
The first theoretical description of such a cryptographically secured chain was made by Stuart Haber and W. Scott Stornett in 1992.\cite{book:haber} In 2008 a person or group known as Satoshi Nakamoto
described the first practical implementation for a blockchain in their white paper called "Bitcoin: A Peer-to-Peer Electronic Cash System".\cite{url:bitcoin} In this paper a new type of digital
currency, later categorised as cryptocurrency, which uses a blockchains for keeping track of transactions is described in detail. One year later this concept has been implemented and released to the
public. The idea behind these blockchains is that they are part of a network of peers. Within this network the blockchain is public knowledge to everyone and at least the majority of the peers in the
network should be in consensus about a newly added block.

\subsection{Byzantine Fault Tolerance}

Byzantine Fault Tolerance is a kind of fault tolerance especially important in distributed systems such as Peer-to-Peer systems but also common in other types of computer systems.
Byzantine Fault Tolerance tries to solve the so called Byzantine Generals' Problem. The Byzantine Generals' Problem is defined as follows: the Byzantine army is split up in multiple divisions 
outside an enemy city. Each division has a leading general which has to decide whether to attack the city or to retreat. The generals communicate to each other via messengers and they all try to
reach a consensus on a plan of action. Some of the generals however may be traitors who try to prevent an agreement. They do this by sending different conflicting messages to the loyal generals. 
Byzantine Fault Tolerance tries to make sure that a small number of such treacherous generals are not able to disrupt a consensus between the majority of the loyal generals.\cite{url:byzantine_general}
Algorithms developed for the purpose of solving this problem in distributed systems are called consensus algorithms.

\subsection{Blockchains in the example of Bitcoin}

The blockchain is a core component of the way Bitcoin and other later developed cryptocurrencies word. The blockchain is used as a public ledger for all transaction which happened in the Bitcoin
network. For this purpose every block in the chain contains among other things a collection of transactions that happened in a time period. By following the blocks in the chain it is possible to
determine exactly where a Bitcoin (or a fraction of a Bitcoin) came from and who it belongs to now. Since this Blockchain is public knowledge there is no dispute over who a Bitcoin belongs to.
One of the most important things then is who should be able to add a new block to the blockchain such that every peer in the network can agree to this new block. In other words how a consensus between
the peers can be reached. In the case of Bitcoin this is done through a Proof-of-Work algorithm.

\subsection{Advantages and use cases of blockchains}

Explanation why blockchains solve some of the problems of other systems and what types of applications this could be useful for.

\subsection{Problems with blockchains}

Description of some problems and disadvantages of blockchains in general, without looking at specific implementations yet.

\subsection{Existing implementations}

Some examples of systems where blockchains are already used.


\section{Proof-of-Work}

The idea behind Proof-of type systems is that whoever should be able to add the next block in the blockchain is chosen by proving the ownership of some kind of resource. This resource could be
something determined by computer hardware, like computing power in Proof-of-Work systems or hard drive capacity in Proof-of-Space systems, or it could be determined by some kind of stake ownership
in the system like in Proof-of-Stake systems. Below we will describe some of these systems and compare them to each other.

\subsection{Proof-of-Work}

The first widely used consensus algorithm in blockchains is a so called Proof-of-Work (POW) algorithm. It was and is still being used as the consensus algorithm in the implementation of Bitcoin.\cite{url:bitcoin}
The idea behind a POW algorithm is that the creator of the next block is chosen by him providing a significant amount of computing power. Usually this computing power is provided by solving some
kind of puzzle. 

\subsubsection{Requirements for POW systems}

The puzzle chosen for a POW system can and should have some properties. Firstly it should be computationally hard enough to ensure that a significant amount of computational power should have 
to be expended to be able to solve it otherwise the purpose would be lost. On the other hand however it should also be computationally easy for others in the system to check the correctness 
of a proposed solution. Additionally it should also be possible to adjust the difficulty of the puzzle. This is very important to make the system future proof and to account for the predicted
exponential increase in computing power over the years.\cite{url:moore_law} Additionally in the context of blockchains the difficulty of the puzzle should also account for a potential
increase in computing power within the network caused by the addition of more peers.\par
Lastly there is also the consideration of usefulness of the puzzle. In most existing implementations
of POW systems the puzzle which should be solved does not provide a higher purpose. In other words the solution of the puzzle has no value other then finding the creator of the next block
and the expended computing power is could be considered to be wasted.\cite{url:pow_useless} There have been some propositions and implementations for "useful" POW type systems. One example is
the algorithm used in the cryptocurrency Primecoin. Primecoin uses a POW type system where a byproduct of the solution for the puzzle is the discovery of potentially new prime numbers.\cite{url:primecoin}

\subsubsection{Examples for existing POW systems}





\subsection{Practical Byzantine Fault Tolerance}

Descritption of the PBFT algorithm from Miguel Castro and Barbara Liskov.

(I wasn't able to read up enough about PBFT yet, so I am not sure about potential additional subsections here).


\subsection{Proof of activity}

Description about POA as a hybrid of POW and POS.


\subsubsection{Problems solved by POA}

Advantages of POA compared to POW and POS.


\subsubsection{Existing implementations and use cases of POA}


\subsection{Proof-of-capacity or Proof-of-space}

Description of what POC and how it works.


\subsubsection{Comparison to POW and POS}

Comparison of POC to POW and POS and advantages/disadvantages


\subsubsection{Existing implementations and use cases of POC}


\subsection{Proof of burn}


\subsubsection{Use cases for POB}


\subsection{Proof of elapsed time}


\subsubsection{Intel's implementation of proof of elapsed time}


\subsubsection{Problems with Proof of elapsed time}


\section{Conclusions}


%+++++++++++++++++++++++++++++++++++++++++++
% edit literature.bib
%\bibliographystyle{abbrv}
\bibliographystyle{unsrt}
\bibliography{literature}

%+++++++++++++++++++++++++++++++++++++++++++
% APPENDICES are optional
% Enable when needed!
% \appendix

% Appendix #1
%\section{Some Appendix Which is Entirely Optional}

\end{document}

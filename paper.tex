% Net.In.Tum (Pro)Seminar Paper Template
% Based on http://www.acm.org/sigs/publications/proceedings-templates
% (Option #1)
% 2015-11-04

% +++++++++++++++++++++++++++++++++++++++++++

% ENGLISH-Version
\documentclass{acm_proc_article-sp}

% GERMAN-Version
%\documentclass{acm_proc_article-sp-german}
%\usepackage[ngerman]{babel}
% Be sure to toggle other language relevant settings accordingly below...

% +++++++++++++++++++++++++++++++++++++++++++
% Include packages for additional functionality

\usepackage{makeidx}
\usepackage{graphicx}
\usepackage{tabularx}
\usepackage{amssymb}
\usepackage{amsmath}
\usepackage{color}
\usepackage{hyperref}
\usepackage[utf8]{inputenc}

%+++++++++++++++++++++++++++++++++++++++++++

\begin{document}

\title{Consensus Algorithms for Blockchains}

\numberofauthors{1}
\author{Marcel Mussner\\
       \affaddr{Advisor: Sree Harsha Totakura}\\
       %\affaddr{Betreuer: Name Vorname}\\
       \affaddr{(Pro)Seminar Network Hacking, Defense and Tamperproof Systems WS2017/18}\\
       \affaddr{Chair for Network Architectures and Services}\\
       \affaddr{Department of Computer Science, Technische Universität München}\\
       % \affaddr{Lehrstuhl für Netzarchitekturen und Netzdienste}\\
       % \affaddr{Fakultät für Informatik, Technische Universität München}\\
       \email{Email: marcel.mussner@tum.de}
}

\maketitle

%+++++++++++++++++++++++++++++++++++++++++++

\begin{abstract}


\end{abstract}

%+++++++++++++++++++++++++++++++++++++++++++

%\keywords{Foo, Bar, Baz}

%+++++++++++++++++++++++++++++++++++++++++++

\section{Introduction}


%+++++++++++++++++++++++++++++++++++++++++++
% Begin of paper's body

\section{Blockchains}
\label{Sec:Something}

Explanation about what a blockchain is and what defines a blockchain. Some historic background (Satoshi's paper about bitcoin). Earlier mentions of similar systems.


\subsection[Technical description of blockchains]

More detailed explanation about how blockchains should work. Short overview about what parts of blockchains are implementation dependent. What differentiates different blockchains.


\subsection{Advantages and use cases of blockchains}

Explanation why blockchains solve some of the problems of other systems and what types of applications this could be useful for.


\subsection{Problems with blockchains}

Description of some problems and disadvantages of blockchains in general, without looking at specific implementations yet.


\subsection{Existing implementations}

Some examples of systems where blockchains are already used.


\section{Consensus algorithms}

Description what a defines a consensus algorithm and why it's a crucial part of how blockchains operate.


\subsection{Requirements for consensus algorithms}

Description what a consensus algorithm should have, also in reference to Byzantine Fault Tolerance.


\subsubsection{Byzantine Fault Tolerance}

Description of what BFT means, what the Byzantine General's problem is and why it is so important in distributed systems.


\subsection{Practical Byzantine Fault Tolerance}

Descritption of the PBFT algorithm from Miguel Castro and Barbara Liskov.

(I wasn't able to read up enough about PBFT yet, so I am not sure about potential additional subsections here).


\subsection{Proof-of-work}

Explanation of proof-of-work as a consensus algorithm. Also explanation why it could be used.


\subsubsection{Hashcash}

First usage of POW in Hashcash. Explanation how it works. (Not sure if I should talk a lot about uses of POW other than in blockchains?)


\subsubsection{POW in blockchains}

Description about how POW can be used in blockchains and how it is already beeing used (cryptocurrency)


\subsubsection{Variants of POW}

Challenge-response vs Solution-verification.


\subsubsection{Problems with POW}

Description of problems which arise by using POW in general and especially in congestion with blockchains and some suggested solutions.


\subsection{Proof-of-stake}

Explanation what Proof of stake means and how it works.


\subsubsection{Advantages over POW}

Explanation about the problems POS tries to solve, that POW has.


\subsubsection{Problems with POS}

Explanation of some of the problems with POS and some solution attempts.


\subsubsection{Existing implementations and use cases of POS}

Some use cases where POS is already beeing used.


\subsection{Proof of activity}

Description about POA as a hybrid of POW and POS.


\subsubsection{Problems solved by POA}

Advantages of POA compared to POW and POS.


\subsubsection{Existing implementations and use cases of POA}


\subsection{Proof-of-capacity or Proof-of-space}

Description of what POC and how it works.


\subsubsection{Comparison to POW and POS}

Comparison of POC to POW and POS and advantages/disadvantages


\subsubsection{Existing implementations and use cases of POC}


\subsection{Proof of burn}


\subsubsection{Use cases for POB}


\subsection{Proof of elapsed time}


\subsubsection{Intel's implementation of proof of elapsed time}


\subsubsection{Problems with Proof of elapsed time}


\section{Conclusions}


%+++++++++++++++++++++++++++++++++++++++++++
% edit literature.bib
\bibliographystyle{abbrv}
\bibliography{literature}

%+++++++++++++++++++++++++++++++++++++++++++
% APPENDICES are optional
% Enable when needed!
% \appendix

% Appendix #1
%\section{Some Appendix Which is Entirely Optional}

\end{document}
